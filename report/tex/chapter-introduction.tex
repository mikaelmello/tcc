
% Utilidade de EONs no cenário atual

Historicamente, o tráfego de internet global cresce de forma exponencial, possuindo uma taxa composta de crescimento anual de 45\% nos anos 2000 \cite{network_evolution_2020} e aproximadamente 30\% nos anos 2010 \cite{cisco2011cisco, cisco2012cisco, cisco2013cisco, cisco2014cisco, cisco2015cisco, cisco2016cisco, cisco2017cisco, cisco2018cisco}. Uma fatia significativa deste crescimento anual se deve ao tráfego de dados em redes móveis, cujas taxas de crescimento anuais na última década tem variado entre 50\% a 60\% e são motivadas pelo crescente número de assinaturas de \textit{smartphones} e o volume de dados consumido por assinatura, este alavancado principalmente pelo crescente consumo de conteúdos de vídeo \cite{ericsson_mobility_report_2020}. Além disso, circunstâncias especiais podem incentivar a população a aumentar seu uso de internet: Em Abril de 2020, a empresa Akamai relatou um crescimento de 30\% do tráfego global em apenas um mês, aproximadamente dez vezes a taxa de crescimento esperada, atribuindo o pico de crescimento às mudanças de estilo de vida causadas pela pandemia do COVID-19 \cite{mckeay_2020}.

A infraestrutura responsável por lidar com tamanho tráfego é composta por redes ópticas que têm sido tradicionalmente \textit{rígidas} e \textit{homogêneas}, isto é, redes baseadas em \acrfull{WDM}, ou multiplexação por divisão de comprimento de onda. Redes baseadas em WDM oferecem a possibilidade de estabelecer conexões com comprimentos de onda fixos e com uma taxa de \textit{bits} fixa, em que os canais são modulados com um formato comum e espaçados por uma distância fixa de 50 GHz \cite{eon_tutorial_2014, jinno_eon_benefits}. O crescimento exponencial de demanda motivou a indústria a focar esforços em aumentar a capacidade destas redes, resultando na evolução das taxas de \textit{bits} permitidas de inicialmente 10 Gb/s para 40 Gb/s e por fim 100 Gb/s \cite{vizcaino_eon_energy}. Entretanto, este tipo de rede conta com alguns problemas:

\begin{itemize}
  \item \textbf{Baixa adaptabilidade}. A flexibilidade destas redes é limitada pela configuração do \textit{hardware}, tornando o processo de atualizar ou modificar a rede para adaptar-se às mudanças de demanda ou de condições de rede desafiador \cite{eon_tutorial_2014}.
  \item \textbf{Baixa eficiência espectral}. O desenho da rede deve garantir que o caminho óptico mais longo (pior caso) seja transmitido com qualidade suficiente. Como os comprimentos de onda são fixos e homogêneos, a maioria das conexões irá possuir comprimentos muito menores do que o pior caso, gerando um problema de ineficiência onde há faixas de comprimentos de onda não utilizadas \cite{jinno_eon_benefits,vizcaino_eon_energy}.
  \item \textbf{Limite de futuros avanços}. Com taxas de bits maiores que 100 Gb/s, melhorias na eficiência espectral ao aumentar o número de \textit{bits} por símbolo se torna cada vez mais difícil devido ao limite de Shannon, além do fato de que aumentar a taxa para além de 100 Gb/s é um desafio por si só \cite{jinno_eon_benefits}.
\end{itemize}

Por estas dificuldades, o conceito de \acrfull{EON}, ou redes ópticas elásticas, foi introduzido como um modo de oferecer uma utilização eficiente dos recursos ópticos disponíveis, sendo capaz de acomodar taxas de \textit{bits} que variam desde algumas dezenas de Gb/s até a magnitude de \textit{terabits} por segundo \cite{eon_tutorial_2014, eon_survey_2012}, além da alocação adaptável de recursos de \textit{hardware} e espectrais de acordo com a demanda do tráfego \cite{jinno_eon_benefits}. Na literatura, os termos "flexível", "elástico", "flexgrid ou flexigrid", "gridless" e "adaptável" são usados intercambiavelmente. Esta alocação flexível é permitida graças ao uso de tecnologias como \acrfull{OFDM}, \acrfull{N-WDM} e \acrfull{OAWG} \cite{eon_tutorial_2014}. EONs têm sido amplamente aceitas como uma das melhores soluções com arquiteturas de rede flexíveis e capazes de alocar recursos de forma flexível \cite{eon_ml_rsa_dl_2019}.

Quanto ao design e otimização de \acrfull{EONs}, um dos principais desafios em seu desenvolvimento se trata da alocação eficiente de recursos. Algoritmos e ferramentas de planejamento de rede convencionais (WDM) não podem ser aplicados devido à natureza flexível das redes. Assim, de modo a aproveitar completamente a flexibilidade disponibilizada pelos avanços de tecnologias na camada física, novos algoritmos de alocação de recursos têm sido explorados \cite{eon_tutorial_2014, eon_allocation_2011, eon_allocation_2011_2, eon_allocation_2016, eon_allocation_2017}. Por exemplo, com flexibilidade apenas no número de subportadoras disponíveis para alocação, devem ser usadas técnicas de roteamento e alocação de espectro, \acrfull{RSA}. Caso haja flexibilidade na seleção do formato de modulação, então técnicas de roteamento e atribuição de espectro com modulação adaptativa, \acrfull{RMLSA}, devem ser escolhidas.

A avaliação de desempenho de sistemas de comunicação óptica é um desafio para os pesquisadores. A dinamicidade e complexidade, especialmente em redes ópticas elásticas, torna inviável uma modelagem analítica precisa e o uso de ambientes reais para medições torna a avaliação bastante custosa, devido principalmente aos equipamentos e ferramentas envolvidos. Assim, a simulação é a alternativa disponível para atividades de teste, validação e avaliação de novos mecanismos de controle para o ambiente de redes ópticas \cite{costa2016ons}.

Diferentes ferramentas de simulação de redes ópticas elásticas foram desenvolvidas para auxiliar os pesquisadores a implementar, testar e analisar novos algoritmos ou soluções de problemas diversos na área. Como por exemplo, \acrfull{ONS} \cite{costa2016ons}, \textit{ElasticO++} \cite{TESSINARI201695}, \acrfull{CEONS} \cite{ceons2015} e \textit{Net2Plan} \cite{net2plan}.

Recentemente, com a crescente popularização do uso de \acrfull{ML}, ou aprendizagem de máquina, na academia, estudos sobre a aplicação de ML para solução de problemas relacionados às EONs também têm se popularizado.

\textit{Machine Learning} é o estudo de algoritmos de computação que se auto-otimizam de acordo com um critério de desempenho, usando dados de exemplo ou a própria experiência \cite{mitchell1997ml, alpaydin2020introduction}. Em sua forma mais básica, o método de ML constitui-se em coletar dados relevantes ao domínio do problema a ser resolvido para serem usados por um algoritmo de aprendizagem, o modelo. Este modelo é definido com parâmetros iniciais que são otimizados automaticamente a partir do consumo dos dados coletados. O objetivo do modelo pode ser tanto preditivo, para realizar predições no futuro sobre dados potencialmente desconhecidos; descritivo, para obter-se conhecimento novo acerca dos dados; ou ambos \cite{alpaydin2020introduction, brief_introduction_to_ml}.

Dentre recentes usos de ML em pesquisas relacionadas às EONs, pode-se citar a pesquisa de Yu et al., que desenvolveu uma estratégia RSA baseada em \acrfull{DL} \cite{eon_ml_rsa_dl_2019}, ou aprendizagem profunda, um subgrupo de ML. Guilherme et al. desenvolveram um modelo de DL capaz de identificar estratégias RSA em EONs com 98\% de acurácia \cite{eon_ml_classifier_2020}. Outras pesquisas acerca do uso de ML em problemas na área de EONs podem ser encontradas em \cite{eon_ml_recent_2019} e \cite{eon_ml_survey_2020}.

Como o crescimento do uso de ML em pesquisas no campo de EONs é recente, simuladores de redes ópticas elásticas disponíveis na literatura ainda não possuem integração com modelos de ML para uso durante as simulações. O presente trabalho visa analisar na literatura atual as bibliotecas de ML capazes de importar modelos pré-treinados, especificamente no campo de aprendizagem profunda como o desenvolvido por Yu et al. \cite{eon_ml_rsa_dl_2019}, no contexto de integrá-las à execução de simulações de EONs.

\section{Problema}
\label{intro-problem}

Devido ao recente crescimento do uso de ML em pesquisas no campo de EONs, simuladores de redes ópticas elásticas disponíveis na literatura ainda não possuem suporte nativo à execução de modelos de ML para uso durante as simulações. Assim, pesquisas que envolvem o uso de ML para a solução de problemas como RSA não possuem um \textit{framework} definido para analisar o desempenho de seus algoritmos ou soluções, sendo necessárioas soluções \textit{ad-hoc} para a análise de resultados.

\section{Objetivos}
\label{intro-goals}

O objetivo geral deste trabalho é possibilitar que pesquisas que envolvem ML em EONs usem simuladores de EONs para a análise, validação e comparação de resultados de modelos desenvolvidos, dado que um dos objetivos de simuladores é fornecer uma \textit{framework} comum para a realização destas atividades.

Este objetivo geral será atingido por meio dos seguintes objetivos específicos:

\begin{itemize}
  \item Estudar as bibliotecas de código existentes na literatura capazes de executar modelos de ML;
  \item Analisar qualitativamente bibliotecas e realizar uma seleção contendo escolhas apropriadas para o caso de uso de integração com simuladores de EONs;
  \item Avaliar de forma quantitativa o desempenho das bibliotecas com métricas e parâmetros pré-definidos;
  \item Analisar os resultados das avaliações e elencar as bibliotecas recomendadas para diferentes casos de uso;
  \item Realizar implementações de integração de ML com o ONS como prova de conceito e contribuição para a comunidade acadêmica.
\end{itemize}

\section{Contribuições}
\label{intro-contributions}

A partir das análises qualitativas e quantitativas das bibliotecas de \textit{machine learning} populares na literatura, este trabalho contribui com diversas recomendações acerca de como adicionar suporte para uso de modelos de aprendizagem de máquina em simuladores de EONs, considerando diversos casos de uso de acordo com as necessidades do usuário.

Adicionalmente, são feitas contribuições ao repositório do simulador ONS, onde são realizadas as seguintes implementações: implementação de funcionalidades que permitem a execução de modelos de forma nativa em Java, com as bibliotecas ONNX Runtime e Deeplearning4j; e implementação de uma funcionalidade que permite a comunicação do ONS com um sistema externo por meio de HTTP, possibilitando a executação de modelos de ML de forma abstrata ao ONS. É esperado que estas implementações sejam a base para futuros avanços na integração entre ML e simulações de EONs.

\section{Organização do trabalho}
\label{intro-org}

O trabalho está organizado em 6 capítulos com os seguintes propósitos:

\begin{itemize}
  \item Capítulo 2 - Introdução de conceitos considerados essenciais para o entendimento do trabalho. Serão descritos o uso de simuladores para pesquisas no campo de redes ópticas elásticas, conceitos de ML usados ao longo do trabalho e a metodologia de avaliação de desempenho utilizada neste trabalho.

  \item Capítulo 3 - Descrição das bibliotecas de ML populares na literatura, sendo explicadas suas principais características e o racional para a consideração delas ou não na análise de desempenho.

  \item Capítulo 4 - Descrição da avaliação de desempenho das bibliotecas selecionadas, como métricas utilizadas, carga de trabalho, a máquina em que os testes foram realizados, entre outros, além da análise dos resultados da avaliação.

  \item Capítulo 5 - Considerando o apresentado nos capítulos 3 e 4, serão descritas as recomendações para o suporte do uso de ML em simuladores de acordo com diferentes casos de uso, e as contribuições feitas para o \acrfull{ONS}.

  \item Capítulo 6 - Serão apresentados os resultados obtidos nas avaliações, as recomendações de futuras implementações e as contribuições realizadas ao \acrfull{ONS}.
\end{itemize}

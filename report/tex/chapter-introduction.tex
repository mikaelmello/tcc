

\section{Problema}
\label{intro-problem}

% Utilidade de EONs no cenário atual



% ONS como ferramenta para pesqiusa sobre EONs

% Pesquisa sobre o uso de ML para EONs

\section{Objetivos}

\section{Metodologia}

\section{Organização do trabalho}

O trabalho está organizado em N \todo capítulos com os seguintes propósitos:

\begin{itemize}
  \item Capítulo 2 - Introdução de conceitos considerados essenciais para o entendimento do trabalho. São descritos os propósitos de redes ópticas elásticas (EONs), do Optical Network Simulator (ONS) \cite{costa2016ons}, conceitos básicos de aprendizagem de máquina, de aprendizagem profundo, e de redes neurais profundas.


  \item Capítulo X \todo - Descrição das tecnologias de aprendizado de máquina e redes neurais profundas popularizadas no mercado, sendo explicadas suas principais características e a motivação para a consideração delas ou não na análise de desempenho.
\end{itemize}


% % Uma introdução é organizada em um conjunto de parágrafos, incluindo: 

% % - contexto (domínio da pesquisa)
% % - problema / motivação da pesquisa (algo que ainda não foi explorado) 
% % - visão geral da contribuição (para resolver esse problema, neste artigo...)

% % - contexto (domínio da pesquisa)

% - Como EON fornece valor
% - Como o simulador fornece valor para pesquisas em EON
% - Como os modelos de ML fornecem valor para o simulador e para casos de uso reais.

% - Modelos de ML no contexto deste TCC são úteis apenas para simulação ou também para uso real?
% - Sim. Seriam. A pesquisa é ver exatamente isso. A ideia é que sejam utilizados em aplicações reais

% Assim, a motivação deste trabalho é propor, no contexto das pesquisas mencionadas acima, uma definição de tecnologias e plataformas que diminuam o tempo de treinamento e de aplicação dos modelos de aprendizado de máquina desenvolvidos pelos pesquisadores.


% % - problema / motivação da pesquisa (algo que ainda não foi explorado) 

% Falar da falta de estrutura de treinamento? Falta de conhecimento sobre a pipeline?
% Falta de otimização dessa pipeline e consequentemente perda de produtividade?

% % - visão geral da metodologia de pesquisa (a proposta foi avaliada usando...) 


% % - visão geral dos principais achados (não precisa incluir nesse momento)


% % - organização do trabalho (o que tem nas próximas seções [opcional])

% Assim, o capítulo \ref{chapter-ml}
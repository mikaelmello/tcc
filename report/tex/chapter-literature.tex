Pela visão de auxiliar pesquisadores em suas integrações de modelos de ML com simuladores de redes ópticas elásticas, as análises aqui feitas devem buscar o aumento de produtividade do pesquisador. Para isto, as tecnologias de ML analisadas serão avaliadas de forma qualitativa, sendo esta avaliação guiada pelas seguintes questões:

\begin{itemize}
  \item Há algum custo para usar a tecnologia?
  \item O código da tecnologia é aberto?
  \item A tecnologia é ativamente mantida por \textit{maintainers} e/ou pela comunidade?
  \item O uso da tecnologia é amplamente documentado?
  \item A instalação e uso da tecnologia é simples?
  \item A tecnologia permite a execução de modelos pré-treinados?
  \item A tecnologia permite a execução de modelos pré-treinados com outras tecnologias?
\end{itemize}

Nesta análise, apenas tecnologias com licenças de código aberto serão consideradas de modo que os pesquisadores tenham livre acesso às recomendações e sejam capazes de manipulá-las em seus projetos, caso seja necessário. Dentre essas, apenas tecnologias disponíveis para uso em programas Java ou Python serão consideradas.

A linguagem Java é amplamente utilizada para a implementação de simuladores de redes ópticas elásticas, como em \cite{costa2016ons}, \cite{ceons2015} e \cite{net2plan}, por isto, a integração de modelos de ML de forma embutida no simulador permite uma integração performática ao excluir-se a necessidade de comunicação com outros serviços ou processos para executar os modelos.

\begin{figure}[h]
  \centering
  \includegraphics[width=1\textwidth]{img/languages-ml.png}
  \caption{Porcentagem de cientistas de dados e desenvolvedores de inteligência artificial que usam ou priorizam cada linguagem \cite{developer_nation_q1_2017}}
  \label{fig:languagesml}
\end{figure}

A linguagem Python é a mais usada e priorizada para desenvolvimento em ML entre trabalhadores da área, como evidenciado na figura \ref{fig:languagesml}. Assim, uma integração de simuladores com um processo ou serviço independente escrito em Python, responsável por executar os modelos, pode ser mais fácil de ser desenvolvida e mantida por um pesquisador de ML na área de EONs, apesar do custo de tempo da execução graças ao tempo necessário para comunicação entre os processos ou serviços.

As tecnologias também devem ser fáceis de serem instaladas, configuradas e manipuladas de acordo com as necessidades de cada pesquisador. Para isto, é fundamental que suas APIs sejam bem documentadas e que a instalação exija o mínimo de modificações de configurações da máquina, externas ao simulador.

Por fim, o uso de ML se expande por diversos problemas relacionados à EONs, de modo que estes problemas podem beneficiar-se de diferentes tipos de aplicações de ML, de acordo com suas características. Assim, é ideal que a tecnologia de ML escolhida para integração com o ML tenha suporte a diferentes tipos de modelos.

\section{OpenCV}

OpenCV (Open Source Computer Vision Library) é uma biblioteca de código aberto voltada para visão computacional e aprendizagem de máquina, construída para fornecer uma infraestrutura comum para aplicações de visão computacional e acelerar o uso de percepção de máquina em produtos comerciais \cite{ml_site_opencv}.

Apesar do foco principal de OpenCV ser visão computacional, a biblioteca possui um módulo de redes neurais profundas e interfaces para as linguagens Python e Java. Adicionalmente, também é possível realizar a importação de modelos serializados em diversos formatos, como Darknet \cite{ml_site_darknet}, Torch7 \cite{ml_site_torch}, ONNX \cite{ml_site_onnx} e TensorFlow \cite{ml_site_tensorflow}.

A API da biblioteca é extensamente documentada, porém com poucos tutoriais sobre o uso do módulo de DNNs. Entretanto, no quesito facilidade de instalação e configuração os resultados foram variados:

Em Python, para a execução de modelos com apenas o uso da CPU a instalação se resume a instalar o pacote \texttt{opencv-python-headless}, a versão sem dependências de bibliotecas de interfaces de usuário gráficas, e está pronto para uso. Porém, se há interesse em utilizar a GPU na execução dos modelos, o processo de instalação se torna bastante complexo, sendo necessário compilar manualmente a biblioteca considerando diversas configurações do ambiente da máquina, de modo que a criação de uma solução generalizada se torne inviável.

Em Java, o processo de instalação é complexo independente do uso ou não de GPU, sendo necessário o mesmo processo de compilação manual do projeto considerando configurações da máquina, não havendo nenhuma integração com gerenciadores de pacotes populares como \textit{Maven} ou \textit{Gradle}.

Assim, pelas dificuldades presentes na instalação, o único uso de OpenCV considerado é o de Python com as execuções sendo realizadas apenas pela CPU da máquina, sem uso de GPU.

\section{PyTorch}

PyTorch é uma biblioteca de ML de código aberto que provê uma plataforma de pesquisas em aprendizagem profunda, oferecendo máxima flexibilidade e velocidade \cite{pytorch_what_is}, sendo considerada a biblioteca de aprendizagem profunda que mais cresce no mundo \cite{course_fast_ai}. Na análise da biblioteca, não foram encontrados métodos nativos de conversão de modelos para serem inferidos com o uso de PyTorch e por este motivo o seu uso foi descartado.

\section{Scikit Learn}

\section{TensorFlow}

\section{TensorFlow Lite}

TensorFlow Lite \cite{ml_site_tensorflow_lite} é uma \textit{framework} de aprendizagem profunda para execução de modelos em dispositivos. Se trata da versão da popular \textit{framework} TensorFlow que é projetada para execuções em dispositivos com menor poder computacional.

A importação de gráficos é limitada apenas a arquivos do tipo TFLite, porém existem ferramentas para realizar a conversão de formatos comuns em Tensorflow, como Keras e SavedModel. Atualmente, a biblioteca não fornece suporte a execução com uso de GPUs NVIDIA \cite{ml_site_tensorflow_lite_gpus}.

Tendo os fatores acima em consideração, a instalação e uso da biblioteca são simples e amplamente documentados para programas Python. Para a instalação, é necessário instalar a versão do pacote específica para a versão do interpretador Python instalado na máquina, sendo possível configurar uma detecção automática. A biblioteca não possui versões para uso em programas Java que não sejam voltados para Android.

Apesar da limitação de importação de modelos e a impossibilidade de uso em programas Java, TensorFlow Lite será avaliado de forma quantitativa pelo seu foco específico de execução rápida de modelos em dispositivos de borda.

\section{ONNX}

\section{DeepLearning4j}

Eclipse DeepLearning4j \cite{ml_site_deeplearning4j} é uma biblioteca de código-aberto para aprendizagem de máquina distribuída, disponível para Java e Scala. Seu desenvolvimento está ativo e é conduzido pela empresa Konduit.

A biblioteca possui ampla documentação da API e diversos tutoriais que explicam os diferentes usos da biblioteca. É possível importar modelos no formato ONNX, HDF5 e TensorFlow, além de seu próprio formato.

A instalação da biblioteca é extremamente simples para uso de CPU e se dá por meio do gerenciador de pacotes Maven, sendo necessário apenas adicionar apenas algumas linhas na configuração do projeto para que seja possível executar o DeepLearning4j. Para uso de GPU, apenas máquinas com placas de vídeo NVIDIA e com CUDA configurado são suportadas. Neste caso, o processo de instalação é quase o mesmo, tendo como diferença o identificador da biblioteca que depende da versão da GPU instalada na máquina.

A biblioteca DeepLearning4j será avaliada em um programa Java com uso de ambas CPU e GPU.

\section{Keras}

\section{Outras tecnologias}


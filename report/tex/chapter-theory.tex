Este capítulo apresenta conceitos básicos fundamentais para o melhor entendimento do trabalho. Na seção \ref{ml} são introduzidos conceitos fundamentais de aprendizagem de máquina e uma breve descrição de suas principais categorias.

São introduzidos detalhes do funcionamento de simulações de EONs e a relevância deles para o trabalho, aprofundados são aprofundados conceitos de machine learning como suas aplicações, seu funcionamento e sua relevância para EONs. Por fim, na seção \ref{ft_dl_run}, são discutidos detalhes da execução de modelos de \textit{deep learning} importantes para o entendimento de considerações feitas neste trabalho.

\section{Aprendizagem de Máquina}
\label{ml}

\textit{Machine Learning} (ML), ou aprendizagem de máquina é o estudo de algoritmos de computação que se auto-otimizam de acordo com um critério de desempenho, usando dados de exemplo ou a própria experiência \cite{mitchell1997ml, alpaydin2020introduction}.

Em sua forma mais básica, o método de ML é a coleta de um grande número de dados do domínio do problema a ser resolvido e o uso deles em um algoritmo de aprendizagem, o modelo. Este modelo é definido com parâmetros iniciais que são otimizados com o consumo do dos dados. O objetivo do modelo pode ser tanto preditivo, para realizar predições no futuro sobre dados potencialmente desconhecidos, descritivo, para obter-se conhecimento novo acerca dos dados, ou ambos. O processo de obter resultados de um modelo é chamado de inferência, sejam eles preditivos ou descritivos.

Há dois grandes desafios no campo de ML: primeiramente, são necessários algoritmos eficientes para o problema de otimização do modelo inicial, de modo que a fase de treinamento seja completada em tempo viável; segundamente, uma vez que um modelo tenha sido aprendido, sua representação e solução algorítimica para a inferência também devem ser eficientes. \cite{alpaydin2020introduction} \cite{brief_introduction_to_ml}

Normalmente, o campo de ML é dividido em três principais categorias: \textit{supervised learning}; \textit{unsupervised learning}; e \textit{reinforcement learning}.

\subsection{Aprendizado supervisionado}

Em inglês, \textit{supervised learning} é a categoria em que o conjunto de dados de entrada (\textit{training set}) possui um mapeamento para o comportamento esperado, rotulado por um "supervisor" \cite{alpaydin2020introduction}. O objetivo é aprender uma regra geral que mapeie os valores de entrada para os respectivos valores esperados de saída. Os valores de saída podem ser contínuos (problemas de regressão) ou discretos (problemas de classificação) \cite{8527529}.

\subsection{Aprendizado não-supervisionado}

Em inglês, \textit{unsupervised learning} é a categoria em que existe apenas o conjunto de dados de entrada e o objetivo é encontrar regularidades presentes nos mesmos \cite{alpaydin2020introduction}. Este tipo de aprendizagem é capaz de desempanhar várias tarefas, porém a mais comum é \textit{clustering} \cite{8527529}.

\textit{Clustering} é o processo de agrupar dados de modo que a similaridade de dados nos grupos (\textit{clusters}) é alta, porém a similaridade de dados entre grupos diferentes é baixa. Esta similaridade é tipicamente expressada como uma função de distância, que depende do tipo de dados presente no conjunto \cite{8527529}.

Dentre os usos de aprendizagem não-supervisionada, pode-se destacar análise de redes sociais, agrupamento de genes e pesquisa de mercado como aplicações bem-sucedidas \cite{8527529}.

\subsection{Aprendizado por reforço}

Em inglês, \textit{reinforcement learning} é a categoria em que o algoritmo de aprendizagem tem como objetivo aprender uma política de ações que maximizem a recompensa em um dado ambiente \cite{alpaydin2020introduction}.

O paradigma de \textit{reinforcement learning} permite que agentes explorem possíveis ações e refinem seu comportamento utilizando apenas uma avaliação, conhecida como recompensa, tendo como objetivo maximizar seu desempenho de longo prazo \cite{8527529}.

Esta técnica é comumente usada em aplicações como robótica, área de finanças como decisões de investimentos e gerenciamento de estoque \cite{8527529}.

\section{Aprendizagem profunda}



\section{Simulações de EONs}%
\label{ft_eon}

Redes ópticas elásticas são consideradas por pesquisadores uma das melhores soluções atuais para lidar com o crescente tráfego global de dados, provendo diversos benefícios ao comparadas com alternativas tradicionais, sendo a principal vantagem a possibilidade de alocar recursos físicos de forma flexível que viabiliza estratégias mais eficientes de alocação. Pelo fato de estratégias tradicionais de alocação de recursos não serem apropriadas para o cenário flexível, diversos estudos para sobre novas estratégias de alocação têm sido desenvolvidos, e devido à complexidade de redes ópticas elásticas, tais estudos utilizam simulações para validar seus resultados. Recentemente, o uso de aprendizagem de máquina como ferramenta para o desenvolvimento de estratégias tem crescido, entretanto simuladores atuais da literatura não possuem suporte para uso de modelos de aprendizagem de máquina em suas implementações. Este trabalho apresenta uma análise qualitativa e quantitativa de bibliotecas de ML populares na literatura com o objetivo de definir direções e recomendações para futuras implementações, além de realizar uma contribuição ao Optical Network Simulator (ONS) com uma implementação inicial de funcionalidades para dar suporte ao uso de modelos de aprendizagem de máquina em simulações.
Redes ópticas elásticas são consideradas por pesquisadores uma das melhores soluções atuais para lidar com o crescente tráfego global de dados, provendo diversos benefícios ao comparadas com alternativas tradicionais. A flexibilidade na alocação de recursos permite que ela seja realizada de forma adaptável e eficiente. Entretanto, algoritmos tradicionais de alocação de recursos não são apropriados para o cenário dinâmico, acarretando em diversos estudos para o desenvolvimento de novas estratégias de alocação e devido à complexidade de redes ópticas elásticas, estes estudos realizam em boa parte simulações para validar seus resultados. Recentemente, o uso de aprendizagem de máquina como ferramenta para o desenvolvimento de estratégias tem crescido, entretanto simuladores atuais da literatura não possuem suporte para uso de modelos de aprendizagem de máquina em suas implementações. Este trabalho apresenta uma análise qualitativa e quantitativa de bibliotecas de ML populares na literatura com o objetivo de definir direções e recomendações para futuras implementações, além de realizar uma contribuição ao Optical Network Simulator (ONS) com uma implementação inicial de funcionalidades para dar suporte ao uso de modelos de aprendizagem de máquina em simulações.
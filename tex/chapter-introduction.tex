%%%%%%%%%%%%%%%%%%%%%%%%%%%%%%%%%%%%%%%%%%%%%%%%%%%%%%%%%%%%%%%%%%%%%%%%%%%%%%%%
%%%%%%%%%%%%%%%%%%%%%%%%%%%%%%%%%%%%%%%%%%%%%%%%%%%%%%%%%%%%%%%%%%%%%%%%%%%%%%%%
%%%%%%%%%%%%%%%%%%%%%%%%%%%%%%%%%%%%%%%%%%%%%%%%%%%%%%%%%%%%%%%%%%%%%%%%%%%%%%%%
\section{Motivação}%

- Como EON fornece valor

- Como o simulador fornece valor para pesquisas em EON

- Como os modelos de ML fornecem valor para o simulador e para casos de uso reais.

- Modelos de ML no contexto deste TCC são úteis apenas para simulação ou também para uso real?
- Sim. Seriam. A pesquisa é ver exatamente isso. A ideia é que sejam utilizados em aplicações reais

Assim, a motivação deste trabalho é propor, no contexto das pesquisas mencionadas acima, uma definição de tecnologias e plataformas que diminuam o tempo de treinamento e de aplicação dos modelos de aprendizado de máquina desenvolvidos pelos pesquisadores.

%%%%%%%%%%%%%%%%%%%%%%%%%%%%%%%%%%%%%%%%%%%%%%%%%%%%%%%%%%%%%%%%%%%%%%%%%%%%%%%%
%%%%%%%%%%%%%%%%%%%%%%%%%%%%%%%%%%%%%%%%%%%%%%%%%%%%%%%%%%%%%%%%%%%%%%%%%%%%%%%%
%%%%%%%%%%%%%%%%%%%%%%%%%%%%%%%%%%%%%%%%%%%%%%%%%%%%%%%%%%%%%%%%%%%%%%%%%%%%%%%%
\section{Problema}%



Falar da falta de estrutura de treinamento? Falta de conhecimento sobre a pipeline?
Falta de otimização dessa pipeline e consequentemente perda de produtividade?

%%%%%%%%%%%%%%%%%%%%%%%%%%%%%%%%%%%%%%%%%%%%%%%%%%%%%%%%%%%%%%%%%%%%%%%%%%%%%%%%
%%%%%%%%%%%%%%%%%%%%%%%%%%%%%%%%%%%%%%%%%%%%%%%%%%%%%%%%%%%%%%%%%%%%%%%%%%%%%%%%
%%%%%%%%%%%%%%%%%%%%%%%%%%%%%%%%%%%%%%%%%%%%%%%%%%%%%%%%%%%%%%%%%%%%%%%%%%%%%%%%
\section{Objetivos}%

O principal objetivo deste trabalho é definir um conjunto de tecnologias e técnicas que resultem em uma maior produtividade dos pesquisadores. Para isso, são definidos os seguintes objetivos específicos:

\begin{itemize}
  \item Melhoria do tempo de execução do treinamento  dos modelos.
  \item Melhoria do tempo de execução da aplicação dos modelos.
\end{itemize}